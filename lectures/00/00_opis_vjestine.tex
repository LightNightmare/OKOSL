\documentclass{beamer}

\usepackage[english]{babel}
\usepackage[utf8]{inputenc}
\usepackage{listings}
\usepackage{datetime}
\usepackage{graphics}
\usepackage{fancybox}
\usepackage{color}
\usepackage[normalem]{ulem}
\usepackage{tikz}
\usetikzlibrary{shapes,arrows}
\usetheme{CambridgeUS}
\usecolortheme{seagull}
% Changing of bullet foreground color not possible if {itemize item}[ball]
\DefineNamedColor{named}{Purple}{cmyk}{0.52,0.97,0,0.55}
\setbeamertemplate{itemize item}[triangle]
\setbeamercolor{title}{fg=Purple}
\setbeamercolor{frametitle}{fg=Purple}
\setbeamercolor{itemize item}{fg=Purple}
\setbeamercolor{section number projected}{bg=Purple,fg=white}
\setbeamercolor{subsection number projected}{bg=Purple}

\renewcommand{\dateseparator}{.}
\newcommand{\todayiso}{\twodigit\day \dateseparator \twodigit\month \dateseparator \the\year}

\title{Osnove korištenja operacijskog sustava Linux}
\subtitle{00. Organizacija vještine}
\author[Tomislav Maričević]{Tomislav Maričević\\{\small Nositelj: dr. sc. Stjepan Groš}}
\institute[FER]{Sveučilište u Zagrebu \\
				Fakultet elektrotehnike i računarstva}
				
\date{\todayiso}

\begin{document}
    %\beamerdefaultoverlayspecification{<+->}
{
\setbeamertemplate{headline}[] % still there but empty
\setbeamertemplate{footline}{}

\begin{frame}
\maketitle
\end{frame}
}

\begin{frame}
\frametitle{Sadržaj}
\tableofcontents
\end{frame}

\section{Predavanja}
\begin{frame}[t]
\frametitle{Predavanja}
\begin{itemize}
	\item Svaka neradna subota, osim u tjednima ispita
	\item A102 - 2 grupe
	\begin{itemize}
		\item (9)10-12 sati
		\item 12-14(15) sati
	\end{itemize}
	\item Promjena grupe samo uz međusobni dogovor
\end{itemize}
\end{frame}

\begin{frame}[t]
\frametitle{Predavanja (2)}
\begin{itemize}
	\item Format predavanja:
	\begin{itemize}
		\item 1. sat - kratko ponavljanje (5 min) + obrada novog gradiva
		\item 2. sat - zadaci za vježbu i primjeri - praktični rad
		\item eventualni 3. sat - usmeno odgovaranje labosa (popodnevnoj grupi je ovo prvi sat)
	\end{itemize}
\end{itemize}
\end{frame}

\begin{frame}[t]
\frametitle{Termini}
\begin{itemize}
	\item 4.10.
	\begin{itemize}
		\item Jutarnja grupa 10-12h
		\item Popodnevna grupa 12-14h
	\end{itemize}
	\item 11.10.
	\begin{itemize}
		\item Jutarnja grupa 10-12h
		\item Popodnevna grupa 12-14h
		\item Zadan zadatak za 1. labos
	\end{itemize}
	\item 18.10.
	\begin{itemize}
		\item Jutarnja grupa 10-12h
		\item Popodnevna grupa 12-14h
	\end{itemize}
	\item 25.10.
	\begin{itemize}
		\item Jutarnja grupa 10-12h
		\item Popodnevna grupa 12-14h
	\end{itemize}
\end{itemize}
\end{frame}

\begin{frame}[t]
\frametitle{Termini (2)}
\begin{itemize}
	\item 8.11.
	\begin{itemize}
		\item Jutarnja grupa 9-12h: 11-12h usmeno odgovaranje prvog labosa
		\item Popodnevna grupa 12-15h: 12-13h usmeno odgovaranje prvog labosa
	\end{itemize}
	\item 15.11.
	\begin{itemize}
		\item Jutarnja grupa 10-12h
		\item Popodnevna grupa 12-14h
		\item Eventualni nadoknadni termin za sve koji ne zadovolje na prvom ispitivanju prvog labosa
		\item Zadan zadatak za 2. labos
	\end{itemize}
	\item 6.12.
	\begin{itemize}
		\item Jutarnja grupa 10-12h
		\item Popodnevna grupa 12-14h
	\end{itemize}
	\item 10.1.
	\begin{itemize}
		\item Jutarnja grupa 10-12h
		\item Popodnevna grupa 12-14h
	\end{itemize}
\end{itemize}
\end{frame}

\begin{frame}[t]
\frametitle{Termini (3)}
\begin{itemize}
	\item 17.1.
	\begin{itemize}
		\item Jutarnja grupa 9-12h: 11-12h usmeno odgovaranje drugog labosa
		\item Popodnevna grupa 12-15h: 12-13h usmeno odgovaranje drugog labosa
	\end{itemize}
	\item 24.1.
	\begin{itemize}
		\item 10-12h
		\item Eventualni nadoknadni termin za sve koji ne zadovolje na prvom ispitivanju drugog labosa
		\item Ostali ne moraju dolaziti
	\end{itemize}
\end{itemize}
\end{frame}

\section{Ocjenjivanje}
\begin{frame}[t]
\frametitle{Ocjenjivanje}
\begin{itemize}
	\item Bodovi:
	\begin{itemize}
		\item Zadaće 8x5  = 40 bodova
		\item Labosi 2x10 = 20 bodova
		\item Vježbe 8x5  = 40 bodova
	\end{itemize}
	\item Uvjeti za prolazak vještine:
	\begin{itemize}
		\item Sve zadaće uspješno predane
		\item Oba labosa predana i uspješno odgovarana
		\item 20 bodova na vježbama
		\item Sve skupa \textbf{\textgreater 50} bodova
	\end{itemize}
\end{itemize}
\end{frame}

\begin{frame}[t]
\frametitle{Vježbe}
\begin{itemize}
	\item Rade se drugi sat predavanja
	\item Praktični primjeri vezani za gradivo obrađeno prvi sat
	\item Prisustvo i aktivnost se boduju s 5 bodova
	\item Da bi se položila vještina, potrebno je biti prisutan na makar pola vježbi
\end{itemize}
\end{frame}

\begin{frame}[t]
\frametitle{Domaće zadaće}
\begin{itemize}
	\item Zadaju se na dan predavanja - objavljene na FERwebu
	\item Predaja do srijede u 23:59 putem maila
	\item Zadaci su slični zadacima za vježbu obrađenima na predavanju
	\item Za svaku zadaću dobiva se povratna informacija o bodovima, kvaliteti izrade i mogućim poboljšanjima
	\item Potrebno je uspješno predati sve zadaće
\end{itemize}
\end{frame}

\begin{frame}[t]
\frametitle{Laboratorijske vježbe}
\begin{itemize}
	\item Dvije u semestru
	\item Zadaju se dosta rano (prije nego je obrađeno sve potrebno gradivo)
	\item Malo teži zadaci gdje se traži razumijevanje i samostalno snalaženje
	\item Odgovaraju se usmeno - za prolazak je potrebno uspješno odgovarati oba labosa
	\item Eventualni nadoknadni termini - za one koji ne zadovolje na prvom terminu
	\begin{itemize}
		\item Detaljnije i strože ispitivanje - ako se ne zadovolji u ovom terminu, pada se vještina
	\end{itemize}
\end{itemize}
\end{frame}

\section{Cilj vještine}
\begin{frame}[t]
\frametitle{Cilj vještine (1)}
\begin{itemize}
	\item Dati pregled mogućnosti Linuxa
	\item Priviknuti studente na rad u konzoli
	\begin{itemize}
		\item rad u konzoli bez osjećaja “neugode”
	\end{itemize}
	\item Udariti temelje na kojima se gradi daljnje znanje
	\begin{itemize}
		\item gdje tražiti problem, kako pronaći rješenje
	\end{itemize}
	\item Dobiti osjećaj za rješavanje onih problema koji nisu eksplicitno pokriveni u vještini
	\item Definirati znanja potrebna za NKOSL :)
\end{itemize}
\end{frame}

\begin{frame}[t]
\frametitle{Cilj vještine (2)}
\begin{itemize}
	\item Potrebno predznanje je minimalno
	\item Vještina je opcionalna
	\begin{itemize}
		\item pretpostavlja se da vještinu slušaju studenti koji žele naučiti
		\item očekuje se angažiranost studenata, prijedlozi, komentari, kritike (konstruktivne), \ldots
	\end{itemize}
	\item Iz istih razloga prepisivanje je strogo zabranjeno!
\end{itemize}
\end{frame}


\section{Radna okolina}
\begin{frame}[t]
\frametitle{Radna okolina}
\begin{itemize}
	\item Predavanja se sastoje od teorije i primjera
	\item Nema strogo definiranih pravila kako isprobavati primjere
	\begin{itemize}
		\item Računala dostupna u laboratoriju 
		\item Dopušteno (i poželjno) korištenje laptopa
		\item Live distribucije na gore navedenom
		\item \ldots
	\end{itemize} 
	\item Jedini uvjet je pokrenut Linux
	\begin{itemize}
		\item Ubuntu distribucija (preferirana, ali nije uvjet)
		\item Gradivo vrijedi za svaki Linux/Unix (većinom, uvijek postoje iznimke)
	\end{itemize}
\end{itemize}
\end{frame}

\section{Literatura}
\begin{frame}[t]
\frametitle{Literatura}
\begin{itemize}
	\item Nakon svakog predavanja na FERweb stranicama predmeta bit će objavljena osnovna i dodatna literatura
	\item Koristite SVU literaturu koja vam je na raspolaganju!
	\begin{itemize}
		\item Knjige, članci, druga predavanja, stripovi, \ldots :)
		\item Internet (najbolji izvor)
		\item wiki, irc, forumi, google 
		\item Jedna od pogodnosti Linuxa je jaka zajednica
	\end{itemize}
	\item Podijelite korisne informacije sa kolegama!
\end{itemize}
\end{frame}

\end{document}
