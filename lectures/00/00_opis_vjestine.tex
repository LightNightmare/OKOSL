\documentclass{beamer}

\usepackage[english]{babel}
\usepackage[utf8]{inputenc}
\usepackage{listings}
\usepackage{datetime}
\usepackage{graphics}
\usepackage{fancybox}
\usepackage{color}
\usepackage[normalem]{ulem}
\usepackage{tikz}
\usetikzlibrary{shapes,arrows}
\usetheme{CambridgeUS}
\usecolortheme{seagull}
% Changing of bullet foreground color not possible if {itemize item}[ball]
\DefineNamedColor{named}{Purple}{cmyk}{0.52,0.97,0,0.55}
\setbeamertemplate{itemize item}[triangle]
\setbeamercolor{title}{fg=Purple}
\setbeamercolor{frametitle}{fg=Purple}
\setbeamercolor{itemize item}{fg=Purple}
\setbeamercolor{section number projected}{bg=Purple,fg=white}
\setbeamercolor{subsection number projected}{bg=Purple}

\renewcommand{\dateseparator}{.}
\newcommand{\todayiso}{\twodigit\day \dateseparator \twodigit\month \dateseparator \the\year}

\title{Osnove korištenja operacijskog sustava Linux}
\subtitle{00. Organizacija vještine}
\author[Goran Cetušić]{Goran Cetušić\\{\small Nositelj: dr. sc. Stjepan Groš}}
\institute[FER]{Sveučilište u Zagrebu \\
				Fakultet elektrotehnike i računarstva}
				
\date{\todayiso}

\begin{document}
    %\beamerdefaultoverlayspecification{<+->}
{
\setbeamertemplate{headline}[] % still there but empty
\setbeamertemplate{footline}{}

\begin{frame}
\maketitle
\end{frame}
}

\begin{frame}
\frametitle{Sadržaj}
\tableofcontents
\end{frame}

\section{Predavanja}
\begin{frame}[t]
\frametitle{Predavanja}
\begin{itemize}
	\item 10 x 2 sata predavanja
	\item A102 - 2 grupe
	\begin{itemize}
		\item 12-14 sati
		\item 14-16 sati
	\end{itemize}
	\item Promjena grupe samo uz međusobni dogovor
	\item Termini predavanja
	\begin{itemize}
		\item 13.10., 20.10., ...
    \item Svaka subota, osim u tjednima ispita
	\end{itemize}
\end{itemize}
\end{frame}

\section{Ocjenjivanje}
\begin{frame}[t]
\frametitle{Ocjenjivanje}
\begin{itemize}
	\item 4 ispita
	\begin{itemize}
		\item 2 ispita po ciklusu
	\end{itemize}
	\item Zadaci koje treba riješiti do kraja semestra
	\item Prolaz:
	\begin{itemize}
		\item $>$ 50\% na svim blicevima
		\item uspješno predani svi zadaci
	\end{itemize}
\end{itemize}
\end{frame}

\section{Cilj vještine}
\begin{frame}[t]
\frametitle{Cilj vještine (1)}
\begin{itemize}
	\item Dati pregled mogućnosti Linuxa
	\item Priviknuti studente na rad u konzoli
	\begin{itemize}
		\item rad u konzoli bez osjećaja “neugode”
	\end{itemize}
	\item Udariti temelje na kojima se gradi daljnje znanje
	\begin{itemize}
		\item gdje tražiti problem, kako pronaći rješenje
	\end{itemize}
	\item Dobiti osjećaj za rješavanje onih problema koji nisu eksplicitno pokriveni u vještini
	\item Definirati znanja potrebna za NKOSL :)
\end{itemize}
\end{frame}

\begin{frame}[t]
\frametitle{Cilj vještine (2)}
\begin{itemize}
	\item Potrebno predznanje je minimalno
	\item Vještina je opcionalna
	\begin{itemize}
		\item pretpostavlja se da vještinu slušaju studenti koji žele naučiti
		\item očekuje se angažiranost studenata, prijedlozi, komentari, kritike (konstruktivne), \ldots
	\end{itemize}
	\item Iz istih razloga prepisivanje je strogo zabranjeno!
	\item Sve je podložno dogovoru, bitno je da naučite!
\end{itemize}
\end{frame}

\section{Teme}
\begin{frame}[t]
\frametitle{Teme(1)}
\begin{itemize}
	\item Uvod
	\begin{itemize}
		\item Operacijski sustavi, distribucije, sustav pomoći
	\end{itemize}
	\item Rad sa datotekama i direktorijima
	\begin{itemize}
		\item Osnovne operacije – kreiranje, brisanje, \ldots
	\end{itemize}
	\item Struktura datotečnog sustava
	\begin{itemize}
		\item Raspored i značenje datoteka
	\end{itemize}
	\item Pregled datoteka
	\begin{itemize}
		\item Tipovi datoteka, sadržaj datoteka, editiranje
	\end{itemize}
	\item Vlasništvo i dozvole
	\begin{itemize}
		\item Korisnici i grupe, manipulacija dozvolama
	\end{itemize}
	\item Pretraživanja, filtri, cjevovodi
	\begin{itemize}
		\item Preusmjeravanje ispisa, filtriranje, najčešće situacije
	\end{itemize}
\end{itemize}
\end{frame}

\begin{frame}[t]
\frametitle{Teme(2)}
\begin{itemize}
	\item Procesi
	\begin{itemize}
		\item Manipulacija procesima i poslovima
	\end{itemize}
	\item Grafičko sučelje
	\begin{itemize}
		\item Desktop okruženja
	\end{itemize}
	\item Ljuska
	\begin{itemize}
		\item Varijable, skripte, skripte Bash ljuske 
	\end{itemize}
	\item Linux u drugim kolegijima
	\begin{itemize}
		\item rad sa gcc i make alatima, uvod u Bash jezik
	\end{itemize}
	\item Zadnje predavanje namijenjeno je za lakše polaganje kolegija na faksu
	\item NAPOMENA: Vještina je za početnike
\end{itemize}
\end{frame}

\section{Radna okolina}
\begin{frame}[t]
\frametitle{Radna okolina}
\begin{itemize}
	\item Predavanja se sastoje od teorije i primjera
	\item Nema strogo definiranih pravila kako isprobavati primjere
	\begin{itemize}
		\item Računala dostupna u laboratoriju (VMware)
		\item Dopušteno (i poželjno) korištenje laptopa
		\item Live distribucije na gore navedenom
		\item \ldots
	\end{itemize} 
	\item Jedini uvjet je pokrenut Linux
	\begin{itemize}
		\item Ubuntu distribucija (preferirana ali nije uvjet)
		\item Gradivo vrijedi za svaki Linux/Unix
	\end{itemize}
	\item Ispiti se pišu na računalu
	\item Pratite obavijesti na stranicama vještine
\end{itemize}
\end{frame}

\section{Literatura}
\begin{frame}[t]
\frametitle{Literatura}
\begin{itemize}
	\item \ldots znanje je znanje, bez obzira otkuda došlo
	\item Koristite SVU literaturu koja vam je na raspolaganju!
	\begin{itemize}
		\item Knjige, članci, druga predavanja, stripovi, \ldots :)
		\item Internet (najbolji izvor)
		\item wiki, irc, forumi, google 
		\item jedna od pogodnosti Linux-a je jaka zajednica
	\end{itemize}
	\item Postoji nekoliko istaknutih web stranica
	\begin{itemize}
		\item www.lwn.net, linuxtoday.com, tldp.org, linux.com...
	\end{itemize}
	\item Na kraju svakog poglavlja priložit će se korisni linkovi
	\item Podijelite korisne informacije sa kolegama
	\item PONAVLJAM: Bitno je da naučite
\end{itemize}
\end{frame}

\end{document}
