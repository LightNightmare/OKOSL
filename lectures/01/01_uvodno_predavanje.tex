\documentclass{beamer}

\usepackage[english]{babel}
\usepackage[utf8]{inputenc}
\usepackage{listings}
\usepackage{datetime}
\usepackage{graphics}
\usepackage{fancybox}
\usepackage{color}
\usepackage[normalem]{ulem}
\usepackage{tikz}
\usepackage{hyperref}
\usetikzlibrary{shapes,arrows}
\usetheme{CambridgeUS}
\usecolortheme{seagull}
% Changing of bullet foreground color not possible if {itemize item}[ball]
\DefineNamedColor{named}{Purple}{cmyk}{0.52,0.97,0,0.55}
\setbeamertemplate{itemize item}[triangle]
\setbeamercolor{title}{fg=Purple}
\setbeamercolor{frametitle}{fg=Purple}
\setbeamercolor{itemize item}{fg=Purple}
\setbeamercolor{section number projected}{bg=Purple,fg=white}
\setbeamercolor{subsection number projected}{bg=Purple}

\renewcommand{\dateseparator}{.}
\newcommand{\todayiso}{\twodigit\day \dateseparator \twodigit\month \dateseparator \the\year}
\newcommand{\shell}[1]{\texttt{#1}}

\title{Osnove korištenja operacijskog sustava Linux}
\subtitle{01. Uvod}
\author[Goran Cetušić]{Goran Cetušić\\{\small Nositelj: dr. sc. Stjepan Groš}}
\institute[FER]{Sveučilište u Zagrebu \\
				Fakultet elektrotehnike i računarstva}
				
\date{\todayiso}

\begin{document}
    %\beamerdefaultoverlayspecification{<+->}
{
\setbeamertemplate{headline}[] % still there but empty
\setbeamertemplate{footline}{}

\begin{frame}
\maketitle
\end{frame}
}

\begin{frame}
\frametitle{Sadržaj}
\tableofcontents
\end{frame}

\section{Osnovni pojmovi iz operacijskih sustava}
\begin{frame}[t]
\frametitle{Operacijski sustav (1)}
\begin{itemize}
	\item Operacijski sustav ima nekoliko primarnih zadaća:
  \begin{itemize}
    \item Pristup uređajima
    \begin{itemize}
      \item Kontrolira svaki zahtjev za pristup uređajima
      \item Određuje koji zahtjev ima prioritet
    \end{itemize}
    \item Upravljanje korisnicima
    \begin{itemize}
      \item Razlikuje zahtjeve korisnika
    \end{itemize}
	\end{itemize}
  \item Ovakav opis skriva puno detalja, ali definira dvije najbitnije uloge
        operacijskog sustava 
\end{itemize}
\end{frame}

\begin{frame}[t]
\frametitle{Operacijski sustav (2)}
\begin{itemize}
  \item Sklopovlje – mnoštvo kompleksnih uređaja
   \begin{itemize}
     \item Pisanje aplikacija za samo jedan je komplicirano
   \end{itemize}
  \item OS preuzima detalje 
  \begin{itemize}
    \item Korisnik (u teoriji) treba znati samo što želi
    \item OS zna kako pristupiti određenom uređaju
  \end{itemize}
  \item Primjer: pisanje podataka na tvrdi disk
  \begin{itemize}
    \item Korisnik/aplikacija uputi zahtjev za brisanje datoteke
    \item OS primi zahtjev i dalje odlučuje što sa njime
  \end{itemize}
\end{itemize}
\end{frame}

\begin{frame}[t]
\frametitle{Operacijski sustav (3)}
\begin{itemize}
  \item OS je posrednik između aplikacije i sklopovlja
  \begin{itemize}
    \item korisnik $\rightarrow$ aplikacija $\rightarrow$ OS 
          $\rightarrow$ uređaj
  \end{itemize}
  \item Aplikacijama nikada nije dopušteno izravno pristupanje uređajima
  \begin{itemize}
    \item Moglo bi doći do kolizije
    \begin{itemize}
      \item Tome služi operacijski sustav
    \end{itemize}
    \item Primjer iznimke je DOS
  \end{itemize}
\end{itemize}
\end{frame}

\begin{frame}[t]
\frametitle{Osnovni pojmovi (1)}
\begin{itemize}
  \item Kernel -- jezgra sustava
  \begin{itemize}
    \item Ono što nazivamo Linux je jezgra
    \begin{itemize}
      \item Linux u širem smislu je jezgra + aplikacije
    \end{itemize}
  \end{itemize}
  \item Vrste operacijskih sustava
  \begin{itemize}
    \item Monolitni vs. mikrokernel bazirani
    \item Opće namjene, rad u stvarnom vremenu, \ldots
  \end{itemize}
  \item Sklopovska podrška izvršavanju OS-a
  \begin{itemize}
    \item Nadzorni način rada, MMU, \ldots
  \end{itemize}
\end{itemize}
\end{frame}

\section{Unix porodica operacijskih sustava}
\begin{frame}[t]
\frametitle{Unix porodica operacijskih sustava (1)}
\begin{itemize}
  \item Većina se grubo može podijeliti u dvije skupine
  \begin{itemize}
    \item Windows bazirani
    \item Unix bazirani
  \end{itemize}
  \item Suprotno očekivanjima, većina su Unix bazirani
  \item Unix – prvi višekorisnički sustav
  \begin{itemize}
    \item Nastao 1970-ih u Bell Labs laboratoriju
    \item Prvotno potpuno besplatan
  \end{itemize}
\end{itemize}
\end{frame} 

\begin{frame}[t]
\frametitle{Unix porodica operacijskih sustava (2)}
\begin{itemize}
  \item Nastaju razne inačice
  \begin{itemize}
    \item AIX, Solaris (OpenSolaris), HP-UX, \ldots
  \end{itemize}
  \item 1983. Unix je komercijaliziran
  \item Dvije glavne grupe
  \begin{itemize}
    \item BSD verzije
    \item System V Release 4 verzije
  \end{itemize}
  \item Linux \textbf{nije} Unix, Unix je zaštićeni znak
  \begin{itemize}
    \item Spada u grupu tzv. unixoida (Linux, *BSD, \ldots)
  \end{itemize}
\end{itemize}
\end{frame}

\section{GNU}
\begin{frame}[t]
\frametitle{GNU (1)}
\begin{itemize}
  \item Pokušaj slobodne reimplementacije Unix operacijskog sustava
  \begin{itemize}
    \item Započet 1983. godine na MIT-u
    \item Richard Stallman
    \item Puno filozofije
  \end{itemize}
  \item Do 1992. godine na raspolaganju sve osim jezgre operacijskog 
        sustava
\end{itemize}
\end{frame}

\begin{frame}[t]
\frametitle{GNU (2)}
\begin{itemize}
  \item GNU Hurd, mikrokernel OS
  \begin{itemize}
    \item Slabo napredovao
    \item Još uvijek nije izdan
  \end{itemize}
  \item 1992. godine Linus Torvalds objavljuje Linux
  \begin{itemize}
    \item ``\ldots kao hobi, dok ne izađe Hurd'' - Torvalds
  \end{itemize}
  \item Linux nadopunjuje GNU
  \begin{itemize}
    \item Linux OS sa GNU sistemskim alatima
  \end{itemize}
  \item Linux vs GNU/Linux
\end{itemize}
\end{frame}

\section{Razvoj Linuxa}
\begin{frame}[t]
\frametitle{Razvoj Linuxa (1)}
\begin{itemize}
  \item Distribuirani razvoj 
  \item Tisuće programera diljem svijeta
  \begin{itemize}
    \item Velik broj nezavisnih, gledano pojedinačno
    \item Još veći pod sponzorstvom
    \begin{itemize}
      \item Google, IBM, Oracle, Intel, \ldots
    \end{itemize}
  \item Pišu se moduli, zakrpe, dokumentacija
  \item Izmjene se predlažu odgovornim osobama
  \end{itemize}
\end{itemize}
\end{frame}

\begin{frame}[t]
\frametitle{Razvoj Linuxa (2)}
\begin{itemize}
  \item Linux je podijeljen na podsustave
  \item Svaki podsustav ima tzv. ``održavatelja'' (engl. 
        \emph{maintainer})
  \begin{itemize}
    \item Održavatelj odlučuje o izmjenama (više-manje)
  \end{itemize}
  \item Linus Torvalds ima najveći autoritet
  \begin{itemize}
    \item Izmjene danas rijetko idu izravno preko njega
  \end{itemize}
  \item Značajni održavatelji
  \begin{itemize}
    \item Andrew Morton
    \item Greg Kroah-Hartman
  \end{itemize}
\end{itemize}
\end{frame}

\begin{frame}[t]
\frametitle{Razvoj Linuxa (3)}
\begin{itemize}
  \item Model razvoja nekada
  \begin{itemize}
    \item Parne verzije stabilne, neparne razvojne
    \begin{itemize}
      \item 1.0, 1.2, 2.0, 2.2, 2.4 stabilne verzije
      \item 1.1, 1.3, 2.1, 2.3, 2.5 nestabilne verzije
    \end{itemize}
  \end{itemize}
  \item Novi model razvoja
  \begin{itemize}
    \item Počinje od verzije 2.6
    \item Stabilne verzije 2.6.x.y
  \end{itemize}
  \item Moguće skinuti sa \url{www.kernel.org}
\end{itemize}
\end{frame}

\section{Licence}
\begin{frame}[t]
\frametitle{Licence (1)}
\begin{itemize}
  \item Licence (EULA -- end-user licence agreement) su vrlo bitne jer
        određuju prava i obaveze korisnika
  \item Niz različitih tipova licenci
  \begin{itemize}
    \item Komercijalne, ``shareware'', otvoreni kod, \ldots
  \end{itemize}
  \item Licence otvorenog koda su bitne za razvoj Linuxa
  \item GPL (1,2,3), LGPL, Apache, BSD
\end{itemize}
\end{frame}

\begin{frame}[t]
\frametitle{Licence (2)}
\begin{itemize}
  \item 1989. GNU izdaje GPLv1
  \item Razvija se pojam otvorenog koda
  \item Postoje razlike, ali generalni principi su:
  \begin{itemize}
    \item Uz binarni program mora biti dostupan i izvorni kod
    \item Izvorni kod pod takvom licencom ne može biti preuzet i 
          licenciran pod drugom licencom
    \item Izvorni kod se (ne)može koristiti u komercijalnim proizvodima
  \end{itemize} 
\end{itemize}
\end{frame}

\section{Distribucije}
\begin{frame}[t]
\frametitle{Distribucije (1)}
\begin{itemize}
  \item Labavo definirano, distribucija je Linux kernel + skup programa
  \item Većina operacijskih sustava i njihovih alata dolaze u kompletu
  \begin{itemize}
    \item 1 izdavač -- 1 operacijski sustav sa 1 ``distribucijom''
    \begin{itemize}
      \item Microsoft/Windows, Apple/Mac OS X, FreeBSD
    \end{itemize}
  \end{itemize}
  \item Linux ima stotine distribucija
  \begin{itemize}
    \item specijalizirane ili opće namjene
  \end{itemize}
\end{itemize}
\end{frame}

\begin{frame}[t]
\frametitle{Distribucije (2)}
\begin{itemize}
  \item Tri najveće grane distribucija
  \begin{itemize}
    \item Debian, Red Hat, Slackware
  \end{itemize}
  \item Najočitije razlike su grafičko sučelje i instalacijski sustav
  \begin{itemize}
    \item GNOME, KDE, Xfce, Awesome, \ldots
    \item apt, yum, pacman, \ldots
  \end{itemize}
  \item Distribucije su konačni proizvodi, operacijski sustavi u 
        najširem smislu
\end{itemize}
\end{frame}

\section{Komunikacija s računalom}
\begin{frame}[t]
\frametitle{Terminal}
\begin{itemize}
  \item ``Uređaj'' koji prima znakove i prikazuje ispis
  \begin{itemize}
    \item Nekada su terminali bili fizički uređaji
    \item Danas su aplikacije koje oponašaju fizičke terminale
  \end{itemize}
  \item Postoji nekoliko emulatora terminala na Unix sustavima
  \begin{itemize}
    \item xterm, rxvt, gnome-terminal, \ldots
  \end{itemize}
  \item Terminali upravljaju unosom i ispisom znakova
  \item Ljuska interpretira značenje znakova
\end{itemize}
\end{frame}

\begin{frame}[t]
\frametitle{Komunikacija sa računalom (1)}
\begin{itemize}
  \item Dva temeljna načina komunikacije
  \begin{itemize}
    \item Kroz grafičko sučelje i putem ljuske/komandne linije
  \end{itemize}
  \item Ljuska je aplikacija(!) koja prihvaća korisnikove naredbe i 
        izvršava ih
  \begin{itemize}
    \item Ljuska olakšava komunikaciju sa sustavom
  \end{itemize}
  \item Ljuska označava spremnost za prihvaćanje naredbi prikazivanjem 
        naredbenog retka (engl. \emph{command prompt})
\end{itemize}
\end{frame}

\begin{frame}[t]
\frametitle{Komunikacija sa računalom (2)}
\begin{itemize}
  \item Kada ljuska pokrene naredbu, čeka da se njeno izvršavanje 
        završi
  \item Za to vrijeme ljuska ne prikazuje narebenu liniju!
  \begin{itemize}
    \item Moguće \textbf{prisilno} zaustaviti naredbu, Ctrl+C
  \end{itemize}
  \item Primjer:
  \begin{itemize}
    \item Pokrenuti naredbu \shell{cat}
  \end{itemize}
\end{itemize}
\end{frame}

\section{Sustav pomoći}
\begin{frame}[t]
\frametitle{Sustav pomoći}
\begin{itemize}
  \item Linux ima sustav pomoći
  \begin{itemize}
    \item Naredba \shell{man}
    \begin{itemize}
      \item Dostupna na svakom Unixu
    \end{itemize}
    \item Naredba \shell{info}
    \begin{itemize}
      \item Dostupna sa GNU alatima
    \end{itemize}
    \item U direktoriju \shell{/usr/share/doc} dosta materijala
  \end{itemize}
  \item Sadrži opise i načine korištenja naredbi, funkcija i 
        konfiguracijskih datoteka
\end{itemize}
\end{frame}

\begin{frame}[t]
\frametitle{Naredba \shell{man} (1)}
\begin{itemize}
  \item Stranice su podijeljene u sekcije:
  \begin{itemize}
    \item 1. sekcija – korisničke naredbe
    \item 2. sekcija – pozivi operacijskog sustava
    \item 3. sekcija – funkcije iz biblioteka
    \item 4. sekcija – opis posebnih datoteka
    \item 5. sekcija – opis formata datoteka
    \item 6. sekcija – igre
    \item 7. sekcija – pregledi, konvencije, razno
    \item 8. sekcija – administrativne naredbe
  \end{itemize}
\end{itemize}
\end{frame}

\begin{frame}[t]
\frametitle{Naredba \shell{man} (2)}
\begin{itemize}
  \item Korištenje naredbe man
  \begin{itemize}
    \item Pregled neke upute
    \begin{itemize}
      \item \shell{man <ime naredbe>}
      \item \shell{man <sekcija> <ime naredbe>}
    \end{itemize}
    \item Pretraživanje stranica
    \begin{itemize}
      \item \shell{man -k <ključna riječ>}
      \item \shell{man -f <ime datoteke>}
    \end{itemize}
  \end{itemize}
  \item Primjer: Pregledavanje upute za naredbu \shell{man}
  \begin{itemize}
    \item \shell{man man}
  \end{itemize}
\end{itemize}
\end{frame}

\begin{frame}[t]
\frametitle{Naredba \shell{man} (3)}
\begin{itemize}
  \item Standardni dijelovi \shell{man} stranice
  \begin{itemize}
    \item NAME -- ime i kratki opis
    \item SYNOPSIS -- mogući načini korištenja
    \item DESCRIPTION -- dulji opis što naredba radi
    \item OPTIONS -- opcije koje naredba prihvaća
    \item ENVIRONMENT -- varijable okruženja (o njima kasnije)
    \item AUTHOR -- autor stranice/naredbe
    \item SEE ALSO -- koje su vezane naredbe
  \end{itemize}
\end{itemize}
\end{frame}

\begin{frame}[t]
\frametitle{Naredba \shell{man} (4)}
\begin{itemize}
  \item Opcionalni dijelovi se stavljaju unutar $[$ i $]$
  \item Ponavljanje se označava sa \ldots
  \item Obvezni dijelovi su potcrtani
  \item Izlazak iz pregledavanja upute
  \begin{itemize}
    \item malo slovo \shell{q}
  \end{itemize}
  \item Zadatak: pronaći sve varijante naredbe/funkcije \shell{printf}
\end{itemize}
\end{frame}

\begin{frame}[t]
\frametitle{Naredba \shell{info} (1)}
\begin{itemize}
  \item Proširena verzija \shell{man} naredbe
  \item Sintaksa naredbe info:
  \begin{itemize}
    \item \shell{info [<opcija menija>]}
  \end{itemize}
  \item Korištenje naredbe \shell{info}
  \begin{itemize}
    \item Izlazak iz info programa tipkom q
    \item Kretanje po stranici strelicama i tipkama PgUp i PgDn
    \item Dokumentacija je organizirana po stranicama
  \end{itemize}
\end{itemize}
\end{frame}

\begin{frame}[t]
\frametitle{Naredba \shell{info} (2)}
\begin{itemize}
  \item Enter (na nekom linku) prelazi na novu stranicu
  \begin{itemize}
    \item Link se prepoznaje po znaku *
    \item Tipke p, n, u
  \end{itemize}
  \item Tipka ? izlistava raspoložive tipke
  \begin{itemize}
    \item  Pronaći kako se izlazi iz dobivene pomoći!
  \end{itemize}
  \item Zadatak
  \begin{itemize}
    \item Pogledati info stranicu naredbe \shell{info}
    \item Pokrenuti naredbu \shell{info} bez parametara
  \end{itemize}
\end{itemize}
\end{frame}

\begin{frame}[t]
\frametitle{Literatura}
\begin{itemize}
  \item \url{http://www.troubleshooters.com/linux/info.htm}
  \item \url{http://www.schweikhardt.net/man_page_howto.html#q3}
  \item \url{http://www.debian.org/doc/debian-policy/ch-docs.html}
  \item \url{http://www.unix.org/what_is_unix/history_timeline.html}
  \item \url{http://distrowatch.com/dwres.php?resource=major}
\end{itemize}
\end{frame}






\end{document}
