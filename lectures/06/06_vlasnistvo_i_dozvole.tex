\documentclass[table,usenames,dvipsnames]{beamer}

\usepackage[english]{babel}
\usepackage[utf8]{inputenc}
\usepackage{listings}
\usepackage{datetime}
\usepackage{graphics}
\usepackage{fancybox}
\usepackage{color}
\usepackage{courier}
\usepackage[normalem]{ulem}
\usepackage{tikz}
\usepackage{verbatim}
\usepackage{multirow}
\usetikzlibrary{shapes,arrows}
\usetheme{CambridgeUS}
\usecolortheme{seagull}
% Changing of bullet foreground color not possible if {itemize item}[ball]
\DefineNamedColor{named}{Purple}{cmyk}{0.52,0.97,0,0.55}
\setbeamertemplate{itemize item}[triangle]
\setbeamercolor{title}{fg=Purple}
\setbeamercolor{frametitle}{fg=Purple}
\setbeamercolor{itemize item}{fg=Purple}
\setbeamercolor{section number projected}{bg=Purple,fg=white}
\setbeamercolor{subsection number projected}{bg=Purple}

\renewcommand{\dateseparator}{.}
\newcommand{\todayiso}{\twodigit\day \dateseparator \twodigit\month \dateseparator \the\year}
\newcommand{\shell}[1]{\texttt{#1}}
\definecolor{LightGray}{gray}{0.9}

\title{Osnove korištenja operacijskog sustava Linux}
\subtitle{06. Vlasništvo i dozvole}
\author[Goran Cetušić]{Goran Cetušić\\{\small Nositelj: dr. sc. Stjepan Groš}}
\institute[FER]{Sveučilište u Zagrebu \\
				Fakultet elektrotehnike i računarstva}
				
\date{\todayiso}

\begin{document}
    %\beamerdefaultoverlayspecification{<+->}
{
\setbeamertemplate{headline}[] % still there but empty
\setbeamertemplate{footline}{}

\begin{frame}
\maketitle
\end{frame}
}

\begin{frame}
\frametitle{Sadržaj}
\tableofcontents
\end{frame}

\section{Općenito o dozvolama}
\begin{frame}[t]
\frametitle{Općenito o dozvolama (1)}
\begin{itemize}
  \item Nužno je osigurati odgovarajuću zaštitu svih korisnika
  \item Zaštita objekta (npr. direktorij, datoteka) se temelji na:
  \begin{itemize}
    \item Objekt je vlasništvo korisnika i grupe
    \begin{itemize}
      \item Njih ispisuje naredba \shell{ls}
    \end{itemize}
    \item Uz objekt vezano je 9 bitova koji definiraju prava svih
          korisnika na taj objekt
    \begin{itemize}
      \item Definiraju tko smije čitati, pisati i 
               izvršavati/pretraživati
    \end{itemize}
  \end{itemize}
\end{itemize}
\end{frame}

\begin{frame}[t]
\frametitle{Općenito o dozvolama (2)}
\begin{itemize}
  \item Tih devet bitova piše se u simboličkom obliku \shell{-rwxrwxrwx}
  \begin{itemize}
    \item Prva grupa od tri bita (rwx) s lijeve strane definira prava za 
          vlasnika (prvi bit je vrsta datoteke)
    \item Druga grupa od tri bita (sredina) definira prava za grupu
    \item Konačno, zadnja tri bita s desne strane definiraju prava za sve 
          ostale
  \end{itemize}
  \item Zadatak
  \begin{itemize}
    \item Izlistati svoj matični direktorij i pogledati vlasnike, grupu i 
          definirana prava
  \end{itemize}
\end{itemize}
\end{frame}

\begin{frame}[t]
\frametitle{Općenito o dozvolama (3)}
\begin{itemize}
  \item Značenja pojedinih bitova su sljedeća:
    \begin{tabular}{l l}
     r &  Dozvoljeno čitanje direktorija/datoteke \\
     w & Dozvoljeno pisanje u direktorij/datoteku \\
     x &  Dozvoljeno izvršavanje datoteka/pretraživanje direktorija
    \end{tabular}
  \item Primjeri
  \begin{itemize}
    \item[] \shell{rwxr-xr-x}
    \item[] \shell{rw-r-{}-r-{}-}
    \item[] \shell{r-{}-r-{}-r-{}-}
  \end{itemize}
\end{itemize}
\end{frame}

\begin{frame}[t]
\frametitle{Općenito o dozvolama (4)}
\begin{itemize}
  \item Način odlučivanja kada korisnik pokušava pristupiti 
        datoteci/direktoriju:
  \begin{itemize}
    \item Je li vlasnik datoteke isti kao korisnik koji pokušava pristupiti
    \begin{itemize}
      \item Ako je, primjeni prva tri bita za odluku
    \end{itemize}
    \item Je li korisnik član grupe koja je vlasnik datoteke
    \begin{itemize}
      \item Ako je, primjeni druga tri bita za odluku
    \end{itemize}
    \item Inače, primjeni zadnja tri bita za odluku
    \item Vlasnik $\Rightarrow$ Grupa $\Rightarrow$ Ostali
  \end{itemize}
\end{itemize}
\end{frame}

\begin{frame}[t]
\frametitle{Općenito o dozvolama (5)}
\begin{itemize}
  \item Primjer
  \begin{itemize}
    \item Neka datoteka je vlasništvo korisnika student1
    \item Grupa group1 je vlasnik datoteke i postavljene su sljedeće 
          zastavice:
    \begin{itemize}
      \item[] \shell{rwxr-x---}
      \item[-] student1 može čitati, pisati i izvršavati datoteku 
      \item[-] student2, član grupe grupa1, može čitati i izvršavati 
               datoteku
      \item[-] student3, koji nije član grupe grupa1, ne može niti čitati 
               niti pisati niti izvršavati datoteku
    \end{itemize}
  \end{itemize}
\end{itemize}
\end{frame}

\begin{frame}[t]
\frametitle{Općenito o dozvolama (6)}
\begin{itemize}
  \item Primjer
  \begin{itemize}
    \item Neka datoteka vlasništvo je korisnika student1
    \item Grupa group1 je vlasnik datoteke i postavljene su sljedeće 
          zastavice:
    \begin{itemize}
      \item[] \shell{---rwxr-x}
      \item[-] student1 ne može niti čitati, niti pisati niti izvršavati 
               datoteku
      \item[-] student2, član grupe grupa1, može čitati, pisati u datoteku
               i izvršavati datoteku
      \item[-] student3, nije član grupe grupa1, može čitati i izvršavati
    \end{itemize}
  \end{itemize}
\end{itemize}
\end{frame}

\section{Promjena dozvola}
\begin{frame}[t]
\frametitle{Promjena dozvola (1)}
\begin{itemize}
  \item Promjena dozvola obavlja se naredbom \shell{chmod}
  \item Sintaksa naredbe je:
  \begin{itemize}
    \item[] \shell{chmod <dozvole> <objekt>}
  \end{itemize}
  \item Dozvole se mogu zadati oktalno i simbolički
  \item Moguće jer rekurzivno mijenjati prava
  \begin{itemize}
    \item[] \shell{chmod -R <dozvole> <objekt>}
  \end{itemize}
\end{itemize}
\end{frame}

%header - violet, prvi - Cadet blue, drugi - gray
\begin{frame}[t]
\frametitle{Promjena dozvola (2)}
\begin{itemize}
  \item Oktalni prikaz dozvola
  \begin{table}[h]
  \rowcolors{2}{White}{LightGray}
  \begin{tabular}{|c|l|l|}
    \hline
    \rowcolor{BlueViolet!20}Oktalna znamenka  & Binarno & Prava       \\
    \hline
    0                 & 000     & \shell{---} \\   
    \hline
    1                 & 001     & \shell{--x} \\
    \hline
    2                 & 010     & \shell{-w-} \\
    \hline
    3                 & 011     & \shell{-wx} \\ 
    \hline
    4                 & 100     & \shell{r--} \\
    \hline
    5                 & 101     & \shell{r-x} \\
    \hline
    6                 & 110     & \shell{rw-} \\
    \hline
    7                 & 111     & \shell{rwx} \\ 
    \hline
  \end{tabular}
  \end{table}
\end{itemize}
\end{frame}

\begin{frame}[t]
\frametitle{Promjena dozvola (3)}
\begin{itemize}
  \item Zadatak
  \begin{itemize}
    \item Kopirati datoteku \shell{/etc/passwd} u svoj matični direktorij
    \item Promijeniti dozvole datoteke tako da isključivo vlasnik ima 
          pravo čitanja i pisanja u datoteku
    \item Provjeriti s naredbom \shell{ls} jesu li dozvole dobro 
          postavljene
    \item Promijeniti dozvole tako da vlasnik, grupa i svi ostali imaju 
          pravo samo čitanja datoteke
    \item Provjeriti naredbom \shell{ls} jesu li dozvole dobro postavljene
    \item Dozvoliti grupi sva prava na datoteku (čitanje, pisanje, 
          izvršavanje), a sebi i ostalima ukloniti sva prava
    \item Pokušati ispisati sadržaj datoteke
  \end{itemize}
\end{itemize}
\end{frame}

\begin{frame}[t]
\frametitle{Promjena dozvola (4)}
  \begin{table}[h]
%  \rowcolors{2}{White}{LightGray}
  \begin{tabular}{|c|l|l|}
    \hline
    \rowcolor{BlueViolet!20}Simbolički prikaz & Vrijednost & Opis       \\
    \hline
    \multirow{4}{*}{Tko} & \multirow{2}{*}{a} & Svi korisnici (vlasnik, 
           vlasnikova \\  & & grupa, svi ostali korisnici) \\ \cline{2-3}
                          & g & Vlasnikova grupa \\ \cline{2-3}
                          & o & Svi ostali korisnici \\ \cline{2-3}
                          & u & Samo vlasnik \\ \hline

    \multirow{3}{*}{Operator} & + & Dodaje mod \\ \cline{2-3}
                              & - & Oduzima mod \\ \cline{2-3}
     & = & Postavlja potpunu vrijednost moda\\ \hline 
    \multirow{3}{*}{Dozvola} & r & Postavlja dozvolu čitanja \\ \cline{2-3}
                             & w & Postavlja dozvolu pisanja \\ \cline{2-3}
     & x & Postavlja dozvolu izvršavanja\\ \hline
  \end{tabular}
  \end{table}
\end{frame}

\begin{frame}[t]
\frametitle{Promjena dozvola (5)}
\begin{itemize}
  \item Primjer (simbolički oblik)
  \begin{itemize}
    \item[] \shell{chmod ugo=rwx file1}
    \begin{itemize}
      \item Vlasnik, grupa i ostali imaju sva dopuštenja
    \end{itemize}
    \item[] \shell{chmod a=rwx file1}
    \begin{itemize}
      \item Svi imaju sva dopuštenja
      \item Identično prethodnoj naredbi
    \end{itemize}
    \item[] \shell{chmod u=rwx,go=rx file1 file2}
    \begin{itemize}
      \item Moguće je kombinirati dopuštenja
      \item Vlasnik može sve dok ostali mogu čitati i izvršavati datoteke
    \end{itemize}
  \end{itemize}
\end{itemize}
\end{frame}
   
\begin{frame}[t]
\frametitle{Promjena dozvola (6)}
\begin{itemize}
  \item Primjer
  \begin{itemize}
    \item[] \shell{chmod g+w file1 file2 file3}
    \begin{itemize}
      \item Dodavanje prava čitanja grupi
    \end{itemize}
    \item[] \shell{chmod -x file1 file2}
    \begin{itemize}
      \item Oduzimanje prava izvršavanja svim korisnicima
    \end{itemize}
  \end{itemize}
  \item Zadatak
  \begin{itemize}
    \item Napraviti datoteku \shell{file1}
    \item Oduzeti grupi i korisnicima koji nisu vlasnici prava čitanja i 
          izvršavanja datoteke
  \end{itemize}
\end{itemize}
\end{frame}

\begin{frame}[t]
\frametitle{Promjena dozvola (7)}
\begin{itemize}
  \item Zadatak
  \begin{itemize}
    \item Kreirati direktorij s nazivom \shell{dir}
    \item Kopirati u taj direktorij datoteku \shell{/etc/passwd}
    \item Izlistati sadržaj direktorija \shell{dir}
    \item Ukloniti si pravo izvršavanja/pretraživanja direktorija i ponovo
          izlistati njegov sadržaj
    \item Ukloniti si pravo čitanja direktorija i pokušati izlistati njegov
          sadržaj
  \end{itemize}
\end{itemize}
\end{frame}

\begin{frame}[t]
\frametitle{Vlasnik datoteke}
\begin{itemize}
  \item Vlasnik datoteke može bez obzira na trenutne dozvole promijeniti 
        dozvole
  \begin{itemize}
    \item Ne može zaobići trenutne dozvole
    \item Ne može promijeniti vlasnika datoteke
    \item Može obrisati datoteku
  \end{itemize}
  \item Zadatak
  \begin{itemize}
    \item Stvoriti datoteku \shell{vlasnik} i postaviti dozvole na 
          \shell{000}
    \item Pokušati obrisati datoteku kao vlasnik datoteke
  \end{itemize}
\end{itemize}
\end{frame}

\begin{frame}[t]
\frametitle{Podrazumijevane dozvole (1)}
\begin{itemize}
  \item Kada se kreira nova datoteka ona prima neke podrazumijevane dozvole
  \begin{itemize}
    \item Na podrazumijevane dozvole utječe se naredbom \shell{umask}
    \item Vrijednost \shell{umask} varijable se XOR-a s vrijednošću 777 i 
          to je nova dozvola datoteke
  \end{itemize}
  \item Primjer: ako je \shell{umask} postavljen na 022 tada će datoteke 
        imati dozvolu 755
  \begin{itemize}
    \item Naredba \shell{umask} bez argumenata ispisuje trenutnu vrijednost
  \end{itemize}
\end{itemize}
\end{frame}
  
\begin{frame}[t]
\frametitle{Podrazumijevane dozvole (2)}
\begin{itemize}
  \item Postavljanje vrijednosti umask obavlja se tako da se zada nova 
        vrijednost (oktalno ili simbolički sa \shell{-S})
  \item Zadatak
  \begin{itemize}
    \item Postaviti umask na 777 i kreirati datoteku \shell{test}
    \item Provjeriti dozvole datoteke \shell{test}
    \item Koju vrijednost umask moramo postaviti kako bi ukinuli sve 
          dozvole grupi i ostalima? 
    \begin{itemize}
      \item Isprobati izračunatu vrijednost
    \end{itemize}
  \end{itemize}
\end{itemize}
\end{frame}

\section{Promjena vlasnika}
\begin{frame}[t]
\frametitle{Promjena vlasnika (1)}
\begin{itemize}
  \item Promjena vlasnika datoteke ili direktorija obavlja se naredbom 
        \shell{chown}
  \item Ta naredba isključivo je dostupna administratoru!
  \item Sintaksa je:
  \begin{itemize}
    \item[] \shell{chown <korisnicko ime> <objekt>}
  \end{itemize}
  \item Opcijom \shell{-R} je moguće rekurzivno postavljanje prava
\end{itemize}
\end{frame}

\begin{frame}[t]
\frametitle{Promjena vlasnika (2)}
\begin{itemize}
  \item Moguće je istovremeno promijeniti korisnika i grupu
  \begin{itemize}
    \item Prvi način
    \item[] \shell{\$ chown <korisnik>:<grupa>  <objekt>}
    \item Drugi način
    \item[] \shell{\$ chown <korisnik>.  <objekt>}
    \begin{itemize}
      \item Točka označava grupu istog imena kao korisnik 
    \end{itemize}
  \end{itemize}
\end{itemize}
\end{frame}

\section{Promjena grupe}
\begin{frame}[t]
\frametitle{Promjena grupe}
\begin{itemize}
  \item Promjena grupe datoteke ili direktorija obavlja se naredbom 
        \shell{chgrp}
  \item Ta naredba isključivo je dostupna administratoru!
  \item Sintaksa je:
  \begin{itemize}
    \item[] \shell{chgrp <ime grupe> <objekt>}
  \end{itemize}
  \item Opcijom \shell{-R} je moguće rekurzivno postavljanje prava 
\end{itemize}
\end{frame}

\section{Naredba sudo}
\begin{frame}[t]
\frametitle{Naredba \shell{sudo} (1)}
\begin{itemize}
  \item engl. \emph{superuser do}
  \item Dozvole ne vrijede ako ste root
  \begin{itemize}
    \item root može sve!!
  \end{itemize}
  \item Naredbom \shell{sudo} privremeno postajete drugi korisnik npr. 
        root korisnik i imate sve ovlasti
  \begin{itemize}
    \item Omogućuje davanje administratorskih ovlasti dodatnim korisnicima
          bez poznavanja lozinke roota
  \end{itemize}
\end{itemize}
\end{frame}

\begin{frame}[t]
\frametitle{Naredba \shell{sudo} (2)}
\begin{itemize}
  \item Sintaksa je
  \begin{itemize}
    \item[] \shell{sudo <opcije> <naredba>}
    \item Opcija \shell{-u} zadaje korisnika pod čijim imenom se izvršava 
          naredba, podrazumijeva se root
  \end{itemize}
  \item Zadatak
  \begin{itemize}
    \item Kao običan korisnik stvorite direktorij naziva \shell{dir1}
    \item Promijenite vlasnika datoteke \shell{dat1} na root
    \item Promijenite dozvole direktorija \shell{dir1} na 644 
  \end{itemize}
\end{itemize}
\end{frame}

\begin{frame}[t]
\frametitle{Datoteka \shell{sudoers} (1)}
\begin{itemize}
  \item Popis korisnika i ovlasti je u \shell{/etc/sudoers}
  \item Sastoji se od dvije vrste zapisa
  \begin{itemize}
    \item Aliasa
    \item Dopuštenja
    \item[] \shell{root\hspace{5pt}    ALL=(ALL) ALL}
    \item[] \shell{cetko   ALL=(ALL) ALL}
  \end{itemize}
  \item Brisanjem root korisnika iz datoteke ograničava roota samo kod 
        korištenja naredbe sudo -- i dalje može sve!
\end{itemize}
\end{frame}

\begin{frame}[t]
\frametitle{Datoteka \shell{sudoers} (2)}
\begin{itemize}
  \item Uvijek editirati naredbom \shell{visudo}!
  \begin{itemize}
    \item \shell{visudo} upozorava na moguće greške
    \item Neke distribucije dopuštaju mijenjanje datoteke samo pomoću 
          \shell{visudo}
  \end{itemize}
  \item Zadatak
  \begin{itemize}
    \item Naredbom \shell{visudo} promijeniti \shell{/etc/sudoers}
    \item Namjerno ubaciti grešku
    \item Pokušati spremiti promjene
  \end{itemize}
\end{itemize}
\end{frame}

\begin{frame}[t]
\frametitle{Datoteka \shell{sudoers} (3)}
\begin{itemize}
  \item Svaka linija označava pravilo
  \begin{itemize}
    \item[] \hspace{12 mm}cetko   ALL=(ALL) ALL
  \end{itemize}
%  \vspace{10 mm}
  \setlength{\unitlength}{0.75mm}
  \begin{picture}(10,10)
    \put(18,0){\vector(1,1){10}}
    \put(43,0){\vector(0,1){10}}
    \put(67,0){\vector(-1,1){10}}
    \put(90,0){\vector(-2,1){20}}
  \end{picture}
  \begin{table}[h]
  \scriptsize
  \raggedright
  \begin{flushleft}
  \begin{tabular}{l l l l}
    Korisnik na kojega & Računalo & Korisnici u čije & Naredbe koje je  \\ 
    se odnosi linija   &          & ime je moguće    & moguće izvršiti  \\
                       &          & izvršiti naredbe &                  \\
  \end{tabular}
  \end{flushleft}
  \end{table}
  \begin{itemize}
    \item Korisnik cetko može na svim računalima kao bilo koji korisnik na            sustavu izvršiti sve naredbe
  \end{itemize}
\end{itemize}
\end{frame}

\begin{frame}[t]
\frametitle{Datoteka \shell{sudoers} (4)}
\begin{itemize}
  \item Primjeri
  \begin{itemize}
    \item[]\footnotesize \shell{okosl ALL=(root) /usr/bin/apt-get, 
           /usr/bin/vim} 
    \item Moguće je definirati grupu
    \item[]\footnotesize \shell{\%sudoers ALL=(root) /usr/bin/apt-get, 
           /usr/bin/vim} 
    \item Ili dopustiti pokretanje bez unošenja lozinke
    \item[]\footnotesize \shell{okosl ALL=(ALL) NOPASSWD: ALL}
  \end{itemize}
\end{itemize}
\end{frame}

\begin{frame}[t]
\frametitle{Datoteka \shell{sudoers} (5)}
\begin{itemize}
  \item Postoje četiri vrste pseudonima (\emph{alias}) za svaki od četiri 
        dijela linije
  \begin{itemize}
    \item Runas\_Alias, User\_Alias, Host\_Alias, Cmnd\_Alias
  \end{itemize}
  \item Primjer
  \begin{itemize}
    \item[] Cmnd\_Alias SHUTDOWN\_CMDS = /sbin/halt, /sbin/reboot
    \item[] User\_Alias USERS = tom, dick, harry, \%admin
    \item[] USERS ALL=(ALL) NOPASSWD: SHUTDOWN\_CMDS
  \end{itemize}
\end{itemize}
\end{frame}

\begin{frame}[t]
\frametitle{Literatura}
\begin{itemize}
  \item \url{http://articles.slicehost.com/2010/7/17/using-chmod-part-1-symbolic-mode}
  \item \url{http://articles.slicehost.com/2010/7/17/using-chmod-part-2-octal-mode}
  \item \url{https://help.ubuntu.com/community/Sudoers}
\end{itemize}
\end{frame}
\end{document}
